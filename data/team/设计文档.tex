\documentclass[a4paper]{article}

\input{style/ch_xelatex.tex}
\input{style/scala.tex}
\graphicspath{ {images/} }
\usepackage{ctex}
\usepackage{graphicx}
\usepackage{color,framed}
\usepackage{listings}
\usepackage{caption}
\usepackage{amssymb}
\usepackage{enumerate}
\usepackage{xcolor}
\usepackage{bm} 
\usepackage{lastpage}
\usepackage{fancyhdr}
\usepackage{tabularx}  
\usepackage{geometry}
\usepackage{float}
\usepackage{multirow}
\usepackage{booktabs}

\geometry{a4paper,left=2.3cm,right=2.3cm,top=2.7cm,bottom=2.7cm}

\usepackage{fancyhdr}
\pagestyle{fancy}
\lhead{\kaishu \leftmark}
\rhead{\kaishu 互联网数据库作业报告}
\lfoot{}
\cfoot{\thepage}
\rfoot{}
\renewcommand{\headrulewidth}{0.1pt}  
\renewcommand{\footrulewidth}{0pt}
\setlength{\textfloatsep}{10mm}

\begin{document}
\renewcommand{\contentsname}{目\ 录}
\renewcommand{\figurename}{图}
\renewcommand{\tablename}{表}
\renewcommand{\today}{\number\year 年 \number\month 月 \number\day 日}

%-------------------------封面----------------
\begin{titlepage}
    \begin{center}
    \includegraphics[width=0.8\textwidth]{NKU.png}\\[1cm]
    \vspace{20mm}
		\textbf{\huge\textbf{\kaishu{计算机学院}}}\\[0.5cm]
		\textbf{\huge{\kaishu{互联网数据库开发}}}\\[2.3cm]
		\textbf{\Huge\textbf{\kaishu{团队作业2——设计文档}}}
		\vspace{\fill}
    
    \centering
    \begin{tabular}{ll}
        \textsc{\LARGE \kaishu{项目名称\ :\ 抗战胜利80周年纪念网站}} \\[0.3cm]
        \textsc{\LARGE \kaishu{团队名称\ :\ 抗战纪念开发组}} \\[0.5cm]
        \textsc{\LARGE \kaishu{组长\ :\ 彭浩然\ (学号:2313314)}} \\[0.3cm]
        \textsc{\LARGE \kaishu{组员\ :\ 刘浩泽\ (学号:2212478)}} \\[0.3cm]
        \textsc{\LARGE \kaishu{组员\ :\ 董珺\ (学号:2212880)}} \\[0.3cm]
        \textsc{\LARGE \kaishu{组员\ :\ 童汉鑫\ (学号:2311995)}} \\[0.3cm]
    \end{tabular}
    \vfill
    {\Large \today}
    \end{center}
\end{titlepage}

\renewcommand {\thefigure}{\thesection{}.\arabic{figure}}
\renewcommand{\figurename}{图}
\renewcommand{\contentsname}{目录}  
\cfoot{\thepage\ of \pageref{LastPage}}

\clearpage
\tableofcontents
\newpage

%--------------------------正文内容--------------------------------

\section{系统架构设计}

\subsection{总体架构}

本系统采用经典的三层架构设计,基于Yii2框架的MVC模式开发,分为前台展示系统和后台管理系统两个子系统。

\subsubsection{架构层次}

\begin{enumerate}
    \item \textbf{表现层(View Layer)}
    \begin{itemize}
        \item 前台:响应式页面,适配PC/平板/手机
        \item 后台:基于Bootstrap的管理界面
        \item 使用HTML5 + CSS3 + JavaScript实现
        \item 采用ECharts进行数据可视化
    \end{itemize}
    
    \item \textbf{业务逻辑层(Controller Layer)}
    \begin{itemize}
        \item 处理用户请求和业务逻辑
        \item 前台控制器:SiteController、GuestbookController等
        \item 后台控制器:ArticleController、UserController、CategoryController等
        \item 实现权限控制和访问验证
    \end{itemize}
    
    \item \textbf{数据访问层(Model Layer)}
    \begin{itemize}
        \item 封装数据库操作
        \item ActiveRecord模式进行ORM映射
        \item 数据验证和业务规则
        \item 模型关联关系定义
    \end{itemize}
    
    \item \textbf{数据层(Database Layer)}
    \begin{itemize}
        \item MySQL数据库
        \item 19张数据表
        \item 存储过程和触发器(可选)
    \end{itemize}
\end{enumerate}

\subsubsection{系统架构图}

系统采用前后端分离的架构设计:

\begin{itemize}
    \item \textbf{前台系统}:面向普通用户,提供内容浏览、互动功能
    \item \textbf{后台系统}:面向管理员,提供内容管理、数据统计功能
    \item \textbf{共享层}:通用模型、配置、组件
\end{itemize}

\subsection{技术架构}

\subsubsection{开发框架}
\begin{itemize}
    \item \textbf{Yii2 Advanced Template}
    \begin{itemize}
        \item 前后台分离的应用模板
        \item 内置用户认证、RBAC权限管理
        \item 强大的数据库查询构建器
        \item 完善的表单验证机制
    \end{itemize}
\end{itemize}

\subsubsection{前端技术栈}
\begin{itemize}
    \item \textbf{HTML5}:语义化标签,提升SEO和可访问性
    \item \textbf{CSS3}:Flexbox布局、Grid布局、动画效果
    \item \textbf{JavaScript (ES6+)}:AJAX异步请求、DOM操作
    \item \textbf{Bootstrap 5}:响应式栅格系统、UI组件
    \item \textbf{ECharts 5}:数据可视化图表库
    \item \textbf{Font Awesome}:图标库
\end{itemize}

\subsubsection{后端技术栈}
\begin{itemize}
    \item \textbf{PHP 8.2}:面向对象编程、命名空间
    \item \textbf{Yii2 Framework}:MVC框架、ActiveRecord ORM
    \item \textbf{Composer}:依赖管理工具
    \item \textbf{MySQL 5.7+}:关系型数据库
\end{itemize}

\subsection{目录结构设计}

\begin{lstlisting}[language=bash, basicstyle=\ttfamily\small]
advanced/
├── backend/                # 后台应用
│   ├── assets/            # 资源包
│   ├── config/            # 配置文件
│   ├── controllers/       # 控制器
│   │   ├── ArticleController.php
│   │   ├── CategoryController.php
│   │   ├── UserController.php
│   │   ├── CommentController.php
│   │   └── GuestBookController.php
│   ├── models/            # 模型
│   ├── views/             # 视图
│   │   ├── article/
│   │   ├── category/
│   │   ├── user/
│   │   └── layouts/
│   └── web/               # Web根目录
│       └── index.php
├── frontend/              # 前台应用
│   ├── assets/            # 资源包
│   ├── config/            # 配置文件
│   ├── controllers/       # 控制器
│   │   ├── SiteController.php
│   │   └── GuestbookController.php
│   ├── models/            # 模型
│   ├── views/             # 视图
│   │   ├── site/
│   │   │   ├── index.php     # 首页
│   │   │   ├── news.php      # 文章列表
│   │   │   ├── view.php      # 文章详情
│   │   │   ├── about.php     # 团队展示
│   │   │   ├── login.php     # 登录
│   │   │   └── signup.php    # 注册
│   │   ├── guestbook/
│   │   └── layouts/
│   └── web/               # Web根目录
│       └── index.php
├── common/                # 共享代码
│   ├── config/            # 共享配置
│   └── models/            # 共享模型
│       ├── PreSysUser.php
│       ├── PreNewsArticle.php
│       ├── PreNewsCategory.php
│       ├── PreNewsComment.php
│       ├── GuestBook.php
│       └── TeamMember.php
├── console/               # 控制台应用
│   └── migrations/        # 数据库迁移
├── data/                  # 数据文件
│   ├── install.sql        # 数据库安装脚本
│   ├── crawler/           # 爬虫脚本
│   ├── team/              # 团队作业文档
│   └── personal/          # 个人作业文档
└── vendor/                # 第三方依赖
\end{lstlisting}

\subsection{模块划分}

\begin{table}[H]
\centering
\caption{系统模块划分表}
\begin{tabularx}{\textwidth}{|l|X|l|}
\hline
\textbf{模块名称} & \textbf{功能描述} & \textbf{负责人} \\ \hline
用户管理模块 & 用户注册、登录、注销、权限控制 & 彭浩然 \\ \hline
文章管理模块 & 文章CRUD、筛选、排序、批量操作 & 刘浩泽 \\ \hline
分类管理模块 & 分类CRUD、分类关联文章管理 & 刘浩泽 \\ \hline
评论管理模块 & 评论发表、审核、删除 & 童汉鑫 \\ \hline
留言管理模块 & 留言发表、查看、已读标记 & 董珺 \\ \hline
团队展示模块 & 团队成员信息展示和管理 & 董珺 \\ \hline
数据统计模块 & 访问统计、内容分布图表 & 刘浩泽 \\ \hline
首页展示模块 & 热门文章、最新文章、时间线 & 刘浩泽 \\ \hline
\end{tabularx}
\end{table}

\section{数据库设计}

\subsection{数据库概述}

\begin{itemize}
    \item \textbf{数据库名称}:news\_system
    \item \textbf{字符集}:utf8mb4
    \item \textbf{排序规则}:utf8mb4\_unicode\_ci
    \item \textbf{存储引擎}:InnoDB
    \item \textbf{数据表数量}:19张(>10张,满足课程要求)
\end{itemize}

\subsection{数据表清单}

\begin{table}[H]
\centering
\caption{数据库表清单}
\begin{tabularx}{\textwidth}{|c|l|X|l|}
\hline
\textbf{序号} & \textbf{表名} & \textbf{说明} & \textbf{设计者} \\ \hline
1 & pre\_sys\_user & 系统用户表 & 彭浩然 \\ \hline
2 & pre\_news\_category & 文章分类表 & 刘浩泽 \\ \hline
3 & pre\_news\_article & 文章主表 & 刘浩泽 \\ \hline
4 & pre\_news\_comment & 文章评论表 & 童汉鑫 \\ \hline
5 & pre\_guestbook & 留言板表 & 董珺 \\ \hline
6 & pre\_article\_tag & 文章标签表 & 刘浩泽 \\ \hline
7 & pre\_article\_tag\_relation & 文章标签关联表 & 刘浩泽 \\ \hline
8 & pre\_article\_view\_log & 文章浏览日志表 & 刘浩泽 \\ \hline
9 & pre\_article\_like & 文章点赞记录表 & 刘浩泽 \\ \hline
10 & pre\_article\_favorite & 文章收藏表 & 刘浩泽 \\ \hline
11 & pre\_article\_attachment & 文章附件表 & 刘浩泽 \\ \hline
12 & pre\_team\_department & 团队部门表 & 董珺 \\ \hline
13 & pre\_team\_member & 团队成员表 & 董珺 \\ \hline
14 & pre\_daily\_stats & 每日统计表 & 刘浩泽 \\ \hline
15 & pre\_category\_stats & 分类统计表 & 刘浩泽 \\ \hline
16 & pre\_link & 友情链接表 & 彭浩然 \\ \hline
17 & pre\_tag & 标签表 & 彭浩然 \\ \hline
18 & pre\_visit\_log & 访问日志表 & 彭浩然 \\ \hline
19 & pre\_config & 系统配置表 & 彭浩然 \\ \hline
\end{tabularx}
\end{table}

\subsection{详细表结构设计}

\subsubsection{用户表(pre\_sys\_user)}

\begin{table}[H]
\centering
\caption{用户表结构}
\begin{tabularx}{\textwidth}{|l|l|l|l|X|}
\hline
\textbf{字段名} & \textbf{类型} & \textbf{空} & \textbf{键} & \textbf{说明} \\ \hline
uid & INT(11) & NO & PRI & 用户ID(主键,自增) \\ \hline
username & VARCHAR(50) & NO & UNI & 用户名(唯一) \\ \hline
password\_hash & VARCHAR(255) & NO & & 密码哈希值 \\ \hline
email & VARCHAR(100) & YES & & 邮箱地址 \\ \hline
phone & VARCHAR(20) & YES & & 手机号码 \\ \hline
avatar & VARCHAR(255) & YES & & 头像URL \\ \hline
role & TINYINT(1) & YES & & 角色:0普通用户 1管理员 \\ \hline
status & TINYINT(1) & YES & & 状态:0禁用 1正常 \\ \hline
created\_at & DATETIME & YES & & 创建时间 \\ \hline
updated\_at & DATETIME & YES & & 更新时间 \\ \hline
\end{tabularx}
\end{table}

\textbf{索引设计:}
\begin{itemize}
    \item PRIMARY KEY: uid
    \item UNIQUE KEY: username
    \item INDEX: email
\end{itemize}

\subsubsection{文章分类表(pre\_news\_category)}

\begin{table}[H]
\centering
\caption{文章分类表结构}
\begin{tabularx}{\textwidth}{|l|l|l|l|X|}
\hline
\textbf{字段名} & \textbf{类型} & \textbf{空} & \textbf{键} & \textbf{说明} \\ \hline
cid & INT(11) & NO & PRI & 分类ID(主键,自增) \\ \hline
name & VARCHAR(50) & NO & UNI & 分类名称(唯一) \\ \hline
description & TEXT & YES & & 分类描述 \\ \hline
sort & INT(11) & YES & & 排序值(越小越靠前) \\ \hline
status & TINYINT(1) & YES & & 状态:0禁用 1启用 \\ \hline
created\_at & DATETIME & YES & & 创建时间 \\ \hline
updated\_at & DATETIME & YES & & 更新时间 \\ \hline
\end{tabularx}
\end{table}

\textbf{预设分类数据:}
\begin{enumerate}
    \item 重大战役
    \item 英雄人物
    \item 历史遗址
    \item 抗战文化
    \item 纪念活动
    \item 历史档案
    \item 老兵故事
    \item 国际视角
\end{enumerate}

\subsubsection{文章主表(pre\_news\_article)}

\begin{table}[H]
\centering
\caption{文章主表结构}
\begin{tabularx}{\textwidth}{|l|l|l|l|X|}
\hline
\textbf{字段名} & \textbf{类型} & \textbf{空} & \textbf{键} & \textbf{说明} \\ \hline
aid & INT(11) & NO & PRI & 文章ID(主键,自增) \\ \hline
cid & INT(11) & NO & FOR & 分类ID(外键) \\ \hline
title & VARCHAR(200) & NO & & 文章标题 \\ \hline
summary & TEXT & YES & & 文章摘要 \\ \hline
content & LONGTEXT & NO & & 文章内容(富文本) \\ \hline
author & VARCHAR(50) & YES & & 作者 \\ \hline
source & VARCHAR(100) & YES & & 来源 \\ \hline
cover\_image & VARCHAR(255) & YES & & 封面图片URL \\ \hline
views & INT(11) & YES & & 浏览量(默认0) \\ \hline
likes & INT(11) & YES & & 点赞数(默认0) \\ \hline
is\_top & TINYINT(1) & YES & & 是否置顶:0否 1是 \\ \hline
is\_hot & TINYINT(1) & YES & & 是否热门:0否 1是 \\ \hline
status & TINYINT(1) & YES & & 状态:0草稿 1已发布 2已下架 \\ \hline
created\_at & DATETIME & YES & & 创建时间 \\ \hline
updated\_at & DATETIME & YES & & 更新时间 \\ \hline
\end{tabularx}
\end{table}

\textbf{索引设计:}
\begin{itemize}
    \item PRIMARY KEY: aid
    \item FOREIGN KEY: cid REFERENCES pre\_news\_category(cid)
    \item INDEX: cid, status, is\_top, is\_hot
    \item INDEX: created\_at, views, likes
\end{itemize}

\subsubsection{评论表(pre\_news\_comment)}

\begin{table}[H]
\centering
\caption{评论表结构}
\begin{tabularx}{\textwidth}{|l|l|l|l|X|}
\hline
\textbf{字段名} & \textbf{类型} & \textbf{空} & \textbf{键} & \textbf{说明} \\ \hline
comment\_id & INT(11) & NO & PRI & 评论ID(主键,自增) \\ \hline
aid & INT(11) & NO & FOR & 文章ID(外键) \\ \hline
uid & INT(11) & NO & FOR & 用户ID(外键) \\ \hline
content & TEXT & NO & & 评论内容 \\ \hline
parent\_id & INT(11) & YES & & 父评论ID(用于回复) \\ \hline
status & TINYINT(1) & YES & & 状态:0待审核 1已通过 2已拒绝 \\ \hline
ip\_address & VARCHAR(50) & YES & & IP地址 \\ \hline
created\_at & DATETIME & YES & & 创建时间 \\ \hline
updated\_at & DATETIME & YES & & 更新时间 \\ \hline
\end{tabularx}
\end{table}

\textbf{索引设计:}
\begin{itemize}
    \item PRIMARY KEY: comment\_id
    \item FOREIGN KEY: aid REFERENCES pre\_news\_article(aid)
    \item FOREIGN KEY: uid REFERENCES pre\_sys\_user(uid)
    \item INDEX: aid, status
\end{itemize}

\subsubsection{留言板表(pre\_guestbook)}

\begin{table}[H]
\centering
\caption{留言板表结构}
\begin{tabularx}{\textwidth}{|l|l|l|l|X|}
\hline
\textbf{字段名} & \textbf{类型} & \textbf{空} & \textbf{键} & \textbf{说明} \\ \hline
id & INT(11) & NO & PRI & 留言ID(主键,自增) \\ \hline
nickname & VARCHAR(50) & NO & & 留言者昵称 \\ \hline
email & VARCHAR(100) & YES & & 邮箱地址 \\ \hline
content & TEXT & NO & & 留言内容 \\ \hline
ip\_address & VARCHAR(50) & YES & & IP地址 \\ \hline
is\_read & TINYINT(1) & YES & & 是否已读:0未读 1已读 \\ \hline
created\_at & DATETIME & YES & & 创建时间 \\ \hline
\end{tabularx}
\end{table}

\subsubsection{统计表设计}

\textbf{每日统计表(pre\_daily\_stats)}

\begin{table}[H]
\centering
\caption{每日统计表结构}
\begin{tabularx}{\textwidth}{|l|l|l|l|X|}
\hline
\textbf{字段名} & \textbf{类型} & \textbf{空} & \textbf{键} & \textbf{说明} \\ \hline
id & INT(11) & NO & PRI & 统计ID(主键,自增) \\ \hline
stat\_date & DATE & NO & UNI & 统计日期(唯一) \\ \hline
total\_views & INT(11) & YES & & 总浏览量 \\ \hline
total\_articles & INT(11) & YES & & 总文章数 \\ \hline
total\_comments & INT(11) & YES & & 总评论数 \\ \hline
created\_at & DATETIME & YES & & 创建时间 \\ \hline
updated\_at & DATETIME & YES & & 更新时间 \\ \hline
\end{tabularx}
\end{table}

\textbf{分类统计表(pre\_category\_stats)}

\begin{table}[H]
\centering
\caption{分类统计表结构}
\begin{tabularx}{\textwidth}{|l|l|l|l|X|}
\hline
\textbf{字段名} & \textbf{类型} & \textbf{空} & \textbf{键} & \textbf{说明} \\ \hline
id & INT(11) & NO & PRI & 统计ID(主键,自增) \\ \hline
cid & INT(11) & NO & UNI & 分类ID(唯一,外键) \\ \hline
article\_count & INT(11) & YES & & 文章数量 \\ \hline
created\_at & DATETIME & YES & & 创建时间 \\ \hline
updated\_at & DATETIME & YES & & 更新时间 \\ \hline
\end{tabularx}
\end{table}

\subsection{数据库E-R图}

\textbf{主要实体关系:}

\begin{enumerate}
    \item \textbf{用户-文章}:一对多关系(一个用户可以发布多篇文章)
    \item \textbf{分类-文章}:一对多关系(一个分类包含多篇文章)
    \item \textbf{文章-评论}:一对多关系(一篇文章可以有多条评论)
    \item \textbf{用户-评论}:一对多关系(一个用户可以发表多条评论)
    \item \textbf{文章-标签}:多对多关系(通过pre\_article\_tag\_relation关联)
    \item \textbf{分类-统计}:一对一关系(每个分类对应一条统计数据)
\end{enumerate}

\subsection{数据完整性约束}

\subsubsection{实体完整性}
\begin{itemize}
    \item 所有表都有主键约束
    \item 主键自增,确保唯一性
\end{itemize}

\subsubsection{参照完整性}
\begin{itemize}
    \item pre\_news\_article.cid 外键引用 pre\_news\_category.cid
    \item pre\_news\_comment.aid 外键引用 pre\_news\_article.aid
    \item pre\_news\_comment.uid 外键引用 pre\_sys\_user.uid
    \item 设置级联操作:ON DELETE CASCADE, ON UPDATE CASCADE
\end{itemize}

\subsubsection{域完整性}
\begin{itemize}
    \item 必填字段设置 NOT NULL 约束
    \item 枚举类型字段使用 TINYINT 存储
    \item 时间字段默认值为 CURRENT\_TIMESTAMP
    \item 计数字段(views, likes)默认值为 0
\end{itemize}

\subsubsection{用户定义完整性}
\begin{itemize}
    \item username 唯一约束
    \item email 格式验证(应用层)
    \item password 强度验证(应用层)
\end{itemize}

\section{接口设计}

\subsection{前台接口设计}

\subsubsection{首页相关接口}

\begin{table}[H]
\centering
\caption{首页接口}
\begin{tabularx}{\textwidth}{|l|l|X|}
\hline
\textbf{URL} & \textbf{方法} & \textbf{功能说明} \\ \hline
/site/index & GET & 获取首页数据(热门文章、最新文章、统计数据) \\ \hline
\end{tabularx}
\end{table}

\textbf{响应数据示例:}
\begin{lstlisting}[language=PHP, basicstyle=\ttfamily\small]
[
    'hotArticles' => [
        ['aid' => 1, 'title' => '...', 'views' => 5000, ...],
        ...
    ],
    'latestArticles' => [...],
    'categories' => [...],
    'dailyStats' => [...],
    'categoryStats' => [...]
]
\end{lstlisting}

\subsubsection{文章相关接口}

\begin{table}[H]
\centering
\caption{文章接口}
\begin{tabularx}{\textwidth}{|l|l|X|}
\hline
\textbf{URL} & \textbf{方法} & \textbf{功能说明} \\ \hline
/site/news & GET & 获取文章列表,支持筛选和排序 \\ \hline
/site/view?id=\{aid\} & GET & 获取文章详情 \\ \hline
/site/like?id=\{aid\} & POST & 文章点赞(AJAX) \\ \hline
\end{tabularx}
\end{table}

\textbf{/site/news 请求参数:}
\begin{table}[H]
\centering
\begin{tabularx}{\textwidth}{|l|l|X|}
\hline
\textbf{参数名} & \textbf{类型} & \textbf{说明} \\ \hline
cid & int & 分类ID(可选) \\ \hline
keyword & string & 搜索关键词(可选) \\ \hline
sort & string & 排序方式:created\_at/views/likes(默认created\_at) \\ \hline
page & int & 页码(默认1) \\ \hline
\end{tabularx}
\end{table}

\textbf{/site/like 响应数据:}
\begin{lstlisting}[language=JSON, basicstyle=\ttfamily\small]
{
    "success": true,
    "message": "点赞成功",
    "likes": 123
}
\end{lstlisting}

\subsubsection{用户认证接口}

\begin{table}[H]
\centering
\caption{用户认证接口}
\begin{tabularx}{\textwidth}{|l|l|X|}
\hline
\textbf{URL} & \textbf{方法} & \textbf{功能说明} \\ \hline
/site/signup & GET & 显示注册页面 \\ \hline
/site/signup & POST & 提交注册表单 \\ \hline
/site/login & GET & 显示登录页面 \\ \hline
/site/login & POST & 提交登录表单 \\ \hline
/site/logout & POST & 用户登出 \\ \hline
\end{tabularx}
\end{table}

\textbf{注册表单数据:}
\begin{lstlisting}[language=PHP, basicstyle=\ttfamily\small]
[
    'username' => 'string',
    'email' => 'string',
    'password' => 'string',
    'password_repeat' => 'string'
]
\end{lstlisting}

\subsubsection{评论接口}

\begin{table}[H]
\centering
\caption{评论接口}
\begin{tabularx}{\textwidth}{|l|l|X|}
\hline
\textbf{URL} & \textbf{方法} & \textbf{功能说明} \\ \hline
/site/comment & POST & 提交评论(需登录) \\ \hline
\end{tabularx}
\end{table}

\textbf{评论表单数据:}
\begin{lstlisting}[language=PHP, basicstyle=\ttfamily\small]
[
    'aid' => 'int',
    'content' => 'string',
    'parent_id' => 'int' // 可选,用于回复
]
\end{lstlisting}

\subsubsection{留言板接口}

\begin{table}[H]
\centering
\caption{留言板接口}
\begin{tabularx}{\textwidth}{|l|l|X|}
\hline
\textbf{URL} & \textbf{方法} & \textbf{功能说明} \\ \hline
/guestbook/index & GET & 显示留言板页面和留言列表 \\ \hline
/guestbook/create & POST & 提交留言 \\ \hline
\end{tabularx}
\end{table}

\subsection{后台接口设计}

\subsubsection{文章管理接口}

\begin{table}[H]
\centering
\caption{后台文章管理接口}
\begin{tabularx}{\textwidth}{|l|l|X|}
\hline
\textbf{URL} & \textbf{方法} & \textbf{功能说明} \\ \hline
/article/index & GET & 文章列表(支持搜索、筛选、排序) \\ \hline
/article/create & GET/POST & 创建文章 \\ \hline
/article/update?id=\{aid\} & GET/POST & 编辑文章 \\ \hline
/article/delete?id=\{aid\} & POST & 删除文章 \\ \hline
/article/view?id=\{aid\} & GET & 查看文章详情 \\ \hline
/article/batch-update & POST & 批量操作(置顶/热门/删除) \\ \hline
/article/toggle-top?id=\{aid\} & POST & 切换置顶状态 \\ \hline
/article/toggle-hot?id=\{aid\} & POST & 切换热门状态 \\ \hline
\end{tabularx}
\end{table}

\textbf{批量操作参数:}
\begin{lstlisting}[language=PHP, basicstyle=\ttfamily\small]
[
    'ids' => [1, 2, 3],
    'action' => 'top' // 或 'untop', 'hot', 'unhot', 'delete'
]
\end{lstlisting}

\subsubsection{分类管理接口}

\begin{table}[H]
\centering
\caption{后台分类管理接口}
\begin{tabularx}{\textwidth}{|l|l|X|}
\hline
\textbf{URL} & \textbf{方法} & \textbf{功能说明} \\ \hline
/category/index & GET & 分类列表 \\ \hline
/category/create & GET/POST & 创建分类 \\ \hline
/category/update?id=\{cid\} & GET/POST & 编辑分类 \\ \hline
/category/delete?id=\{cid\} & POST & 删除分类 \\ \hline
/category/view?id=\{cid\} & GET & 查看分类详情 \\ \hline
\end{tabularx}
\end{table}

\subsubsection{用户管理接口}

\begin{table}[H]
\centering
\caption{后台用户管理接口}
\begin{tabularx}{\textwidth}{|l|l|X|}
\hline
\textbf{URL} & \textbf{方法} & \textbf{功能说明} \\ \hline
/user/index & GET & 用户列表 \\ \hline
/user/create & GET/POST & 创建用户 \\ \hline
/user/update?uid=\{uid\} & GET/POST & 编辑用户 \\ \hline
/user/delete?uid=\{uid\} & POST & 删除用户 \\ \hline
/user/view?uid=\{uid\} & GET & 查看用户详情 \\ \hline
\end{tabularx}
\end{table}

\subsubsection{评论管理接口}

\begin{table}[H]
\centering
\caption{后台评论管理接口}
\begin{tabularx}{\textwidth}{|l|l|X|}
\hline
\textbf{URL} & \textbf{方法} & \textbf{功能说明} \\ \hline
/comment/index & GET & 评论列表 \\ \hline
/comment/approve?id=\{comment\_id\} & POST & 通过评论 \\ \hline
/comment/reject?id=\{comment\_id\} & POST & 拒绝评论 \\ \hline
/comment/delete?id=\{comment\_id\} & POST & 删除评论 \\ \hline
/comment/batch-approve & POST & 批量通过 \\ \hline
\end{tabularx}
\end{table}

\subsubsection{留言管理接口}

\begin{table}[H]
\centering
\caption{后台留言管理接口}
\begin{tabularx}{\textwidth}{|l|l|X|}
\hline
\textbf{URL} & \textbf{方法} & \textbf{功能说明} \\ \hline
/guest-book/index & GET & 留言列表 \\ \hline
/guest-book/view?id=\{id\} & GET & 查看留言详情 \\ \hline
/guest-book/delete?id=\{id\} & POST & 删除留言 \\ \hline
/guest-book/mark-read?id=\{id\} & POST & 标记已读 \\ \hline
\end{tabularx}
\end{table}

\subsection{安全设计}

\subsubsection{CSRF防护}
\begin{itemize}
    \item 所有POST请求需要包含CSRF Token
    \item Yii2自动生成和验证CSRF Token
    \item Token存储在Meta标签和表单隐藏字段中
\end{itemize}

\subsubsection{XSS防护}
\begin{itemize}
    \item 输出时使用Html::encode()转义
    \item 富文本编辑器使用HtmlPurifier过滤
    \item 严格验证用户输入
\end{itemize}

\subsubsection{SQL注入防护}
\begin{itemize}
    \item 使用参数化查询(Prepared Statements)
    \item 通过ActiveRecord进行数据库操作
    \item 避免直接拼接SQL语句
\end{itemize}

\subsubsection{权限控制}
\begin{itemize}
    \item 后台操作需要管理员权限
    \item 使用Yii2的access control filter
    \item Session管理,防止会话劫持
\end{itemize}

\section{界面设计}

\subsection{前台界面设计}

\subsubsection{设计风格}
\begin{itemize}
    \item \textbf{主题色}:深红色(\#8B0000)、金色(\#FFD700)
    \item \textbf{辅助色}:深褐色(\#4a0000)、米色(\#F5F5DC)
    \item \textbf{字体}:微软雅黑、思源黑体
    \item \textbf{布局}:响应式栅格布局,适配PC/平板/手机
    \item \textbf{风格}:庄重、简洁、现代化
\end{itemize}

\subsubsection{导航栏设计}
\begin{itemize}
    \item Logo:网站名称 + 图标
    \item 导航菜单:首页、新闻资讯、团队介绍、留言板
    \item 用户菜单:未登录显示"登录/注册",已登录显示用户名和"退出"
    \item 固定在页面顶部,滚动时保持可见
\end{itemize}

\subsubsection{首页布局}
\begin{enumerate}
    \item \textbf{Hero Banner}
    \begin{itemize}
        \item 全屏宽度,深红渐变背景
        \item 主标题:"铭记历史 珍爱和平"
        \item 副标题:"纪念中国人民抗日战争暨世界反法西斯战争胜利80周年"
        \item 年份标识:1945-2025
    \end{itemize}
    
    \item \textbf{纪念标语}
    \begin{itemize}
        \item 居中显示:"勿忘国耻 · 振兴中华 · 缅怀先烈 · 开创未来"
        \item 金色五角星装饰
    \end{itemize}
    
    \item \textbf{分类导航}
    \begin{itemize}
        \item 横向滚动的标签栏
        \item 显示所有文章分类
        \item 点击跳转到对应分类的文章列表
    \end{itemize}
    
    \item \textbf{热门专题}
    \begin{itemize}
        \item 卡片式布局,3列网格(响应式)
        \item 每个卡片显示封面图、标题、摘要、分类、统计数据
        \item Hover效果:阴影加深、轻微放大
    \end{itemize}
    
    \item \textbf{最新发布}
    \begin{itemize}
        \item 布局同热门专题
    \end{itemize}
    
    \item \textbf{抗战历史时间线}
    \begin{itemize}
        \item 垂直时间轴
        \item 每个节点显示日期、标题、描述
        \item 使用深红色圆点标记
        \item 最后节点(抗战胜利)金色高亮
    \end{itemize}
    
    \item \textbf{数据统计图表}
    \begin{itemize}
        \item 两列布局
        \item 左侧:近30天访问趋势折线图
        \item 右侧:分类内容分布饼图
        \item 使用ECharts渲染
    \end{itemize}
\end{enumerate}

\subsubsection{文章列表页设计}
\begin{itemize}
    \item 顶部横幅:背景图 + 标题
    \item 筛选栏:分类下拉、排序选项
    \item 文章卡片:3列网格(响应式)
    \item 底部分页导航
\end{itemize}

\subsubsection{文章详情页设计}
\begin{itemize}
    \item \textbf{顶部}:阅读进度条(固定在页面最顶部)
    \item \textbf{文章头部}:标题、发布时间、作者、分类、统计数据
    \item \textbf{文章正文}:富文本内容,段落间距适中
    \item \textbf{浮动工具栏}(右侧):
    \begin{itemize}
        \item 点赞按钮(心形图标)
        \item 分享按钮(分享图标)
        \item 返回顶部按钮(箭头图标)
    \end{itemize}
    \item \textbf{相关文章推荐}:底部横向卡片列表
    \item \textbf{评论区}:
    \begin{itemize}
        \item 评论输入框(需登录)
        \item 评论列表(时间倒序)
        \item 每条评论显示:用户头像、昵称、时间、内容
    \end{itemize}
\end{itemize}

\subsubsection{团队展示页设计}
\begin{itemize}
    \item \textbf{团队简介}:项目介绍、开发目标
    \item \textbf{成员展示}:卡片式布局,4列网格
    \item 每个卡片显示:照片、姓名、角色、负责模块
    \item \textbf{分工表格}:详细的分工和工作量说明
\end{itemize}

\subsubsection{留言板页设计}
\begin{itemize}
    \item \textbf{留言表单}:昵称、邮箱(可选)、留言内容
    \item \textbf{留言列表}:卡片式布局
    \item 每条留言显示:昵称、时间、内容
    \item 支持分页显示
\end{itemize}

\subsection{后台界面设计}

\subsubsection{设计风格}
\begin{itemize}
    \item \textbf{主题色}:蓝色(Bootstrap默认主题)
    \item \textbf{布局}:左侧菜单 + 右侧内容区
    \item \textbf{风格}:简洁、高效、功能优先
\end{itemize}

\subsubsection{菜单结构}
\begin{enumerate}
    \item 首页(仪表盘)
    \item 内容管理
    \begin{itemize}
        \item 文章管理
        \item 分类管理
    \end{itemize}
    \item 互动管理
    \begin{itemize}
        \item 评论管理
        \item 留言管理
    \end{itemize}
    \item 系统管理
    \begin{itemize}
        \item 用户管理
        \item 团队管理
    \end{itemize}
\end{enumerate}

\subsubsection{列表页设计}
\begin{itemize}
    \item \textbf{顶部}:页面标题 + 操作按钮(如"添加")
    \item \textbf{筛选栏}:搜索框、下拉筛选、排序选项
    \item \textbf{数据表格}:使用Yii2 GridView组件
    \begin{itemize}
        \item 序号列
        \item 数据列(可排序)
        \item 操作列(查看、编辑、删除)
    \end{itemize}
    \item \textbf{批量操作}:复选框 + 批量操作按钮
    \item \textbf{分页}:底部分页导航
\end{itemize}

\subsubsection{表单页设计}
\begin{itemize}
    \item \textbf{顶部}:页面标题 + 面包屑导航
    \item \textbf{表单区域}:使用Yii2 ActiveForm组件
    \begin{itemize}
        \item 标签在输入框上方
        \item 必填字段标记红色星号
        \item 实时验证提示
    \end{itemize}
    \item \textbf{底部按钮}:提交、取消
\end{itemize}

\section{安全性设计}

\subsection{认证与授权}

\subsubsection{用户认证}
\begin{itemize}
    \item 使用Yii2的User组件管理用户认证
    \item 密码使用bcrypt算法加密(Yii2 Security组件)
    \item Session超时时间:2小时
    \item 支持"记住我"功能(30天)
\end{itemize}

\subsubsection{权限控制}
\begin{itemize}
    \item 前台:
    \begin{itemize}
        \item 游客:可浏览所有公开内容
        \item 注册用户:可点赞、评论
    \end{itemize}
    \item 后台:
    \begin{itemize}
        \item 仅管理员可访问
        \item 使用access control filter拦截未授权访问
    \end{itemize}
\end{itemize}

\subsection{数据安全}

\subsubsection{输入验证}
\begin{itemize}
    \item 前端:HTML5表单验证、JavaScript验证
    \item 后端:Yii2 Model rules验证(必填、格式、长度等)
    \item 富文本内容:HtmlPurifier过滤
\end{itemize}

\subsubsection{SQL注入防护}
\begin{itemize}
    \item 使用ActiveRecord进行数据库操作
    \item 动态条件使用参数绑定
    \item 禁止拼接原始SQL
\end{itemize}

\subsubsection{XSS防护}
\begin{itemize}
    \item 输出时使用Html::encode()
    \item 富文本内容使用HtmlPurifier过滤
    \item CSP(Content Security Policy)头部设置
\end{itemize}

\subsubsection{CSRF防护}
\begin{itemize}
    \item Yii2自动生成CSRF Token
    \item 所有POST请求验证Token
    \item AJAX请求在Header中携带Token
\end{itemize}

\subsection{会话安全}

\begin{itemize}
    \item Session ID使用httponly和secure标志
    \item Session超时自动注销
    \item 登录后重新生成Session ID
    \item 防止Session固定攻击
\end{itemize}

\section{性能优化设计}

\subsection{数据库优化}

\begin{itemize}
    \item \textbf{索引优化}:为常用查询字段建立索引
    \item \textbf{查询优化}:
    \begin{itemize}
        \item 使用预加载(eager loading)减少N+1查询
        \item 分页查询减少数据量
        \item 避免SELECT *,只查询需要的字段
    \end{itemize}
    \item \textbf{缓存优化}:
    \begin{itemize}
        \item 使用Yii2的数据缓存(文件缓存)
        \item 分类列表缓存(1小时)
        \item 统计数据缓存(30分钟)
    \end{itemize}
\end{itemize}

\subsection{前端优化}

\begin{itemize}
    \item \textbf{资源压缩}:CSS/JS文件压缩合并
    \item \textbf{图片优化}:
    \begin{itemize}
        \item 使用WebP格式
        \item 图片懒加载
        \item 响应式图片
    \end{itemize}
    \item \textbf{CDN加速}:
    \begin{itemize}
        \item ECharts、Font Awesome使用CDN
        \item 静态资源CDN分发
    \end{itemize}
    \item \textbf{浏览器缓存}:设置合理的Cache-Control头
\end{itemize}

\subsection{代码优化}

\begin{itemize}
    \item 避免重复代码,提取公共方法
    \item 使用异步加载(AJAX)提升交互体验
    \item 合理使用Yii2的延迟加载
\end{itemize}

\section{测试设计}

\subsection{测试策略}

\begin{enumerate}
    \item \textbf{单元测试}:测试Model层的业务逻辑
    \item \textbf{功能测试}:测试Controller层的功能实现
    \item \textbf{集成测试}:测试前后端集成、数据库交互
    \item \textbf{安全测试}:SQL注入、XSS、CSRF测试
    \item \textbf{性能测试}:并发访问、响应时间测试
    \item \textbf{兼容性测试}:多浏览器、多设备测试
\end{enumerate}

\subsection{测试用例}

\subsubsection{用户认证测试}
\begin{table}[H]
\centering
\caption{用户认证测试用例}
\begin{tabularx}{\textwidth}{|l|X|X|}
\hline
\textbf{测试项} & \textbf{测试步骤} & \textbf{预期结果} \\ \hline
用户注册 & 填写有效信息并提交 & 注册成功,自动登录 \\ \hline
重复用户名 & 使用已存在的用户名注册 & 提示用户名已存在 \\ \hline
密码强度 & 输入弱密码 & 提示密码强度不足 \\ \hline
用户登录 & 输入正确的用户名和密码 & 登录成功,跳转到首页 \\ \hline
错误密码 & 输入错误的密码 & 提示用户名或密码错误 \\ \hline
\end{tabularx}
\end{table}

\subsubsection{文章管理测试}
\begin{table}[H]
\centering
\caption{文章管理测试用例}
\begin{tabularx}{\textwidth}{|l|X|X|}
\hline
\textbf{测试项} & \textbf{测试步骤} & \textbf{预期结果} \\ \hline
创建文章 & 填写完整信息并保存 & 文章创建成功 \\ \hline
必填验证 & 不填写标题直接提交 & 提示标题不能为空 \\ \hline
编辑文章 & 修改文章内容并保存 & 更新成功 \\ \hline
删除文章 & 点击删除并确认 & 文章删除成功 \\ \hline
批量置顶 & 选择多篇文章批量置顶 & 选中文章置顶成功 \\ \hline
\end{tabularx}
\end{table}

\subsubsection{安全测试}
\begin{table}[H]
\centering
\caption{安全测试用例}
\begin{tabularx}{\textwidth}{|l|X|X|}
\hline
\textbf{测试项} & \textbf{测试方法} & \textbf{预期结果} \\ \hline
SQL注入 & 在输入框中输入SQL语句 & 被转义或过滤,不执行 \\ \hline
XSS攻击 & 在评论中输入脚本代码 & 被转义,不执行 \\ \hline
CSRF攻击 & 伪造POST请求(无Token) & 请求被拒绝 \\ \hline
未授权访问 & 游客访问后台管理页面 & 跳转到登录页 \\ \hline
\end{tabularx}
\end{table}

\section{部署设计}

\subsection{部署环境}

\begin{itemize}
    \item \textbf{开发环境}:Windows + XAMPP
    \item \textbf{测试环境}:Linux + Apache + MySQL
    \item \textbf{生产环境}:Linux + Nginx + MySQL(或Apache)
\end{itemize}

\subsection{部署步骤}

\begin{enumerate}
    \item \textbf{环境准备}
    \begin{itemize}
        \item 安装PHP 8.0+
        \item 安装MySQL 5.7+
        \item 安装Composer
        \item 配置Web服务器(Apache/Nginx)
    \end{itemize}
    
    \item \textbf{代码部署}
    \begin{itemize}
        \item 从Git仓库拉取代码
        \item 运行composer install安装依赖
        \item 配置数据库连接(common/config/main-local.php)
    \end{itemize}
    
    \item \textbf{数据库初始化}
    \begin{itemize}
        \item 创建数据库:CREATE DATABASE news\_system
        \item 导入表结构:mysql < data/install.sql
        \item 导入测试数据(可选)
    \end{itemize}
    
    \item \textbf{权限设置}
    \begin{itemize}
        \item 设置runtime目录可写权限
        \item 设置assets目录可写权限
        \item 设置uploads目录可写权限(如有)
    \end{itemize}
    
    \item \textbf{Web服务器配置}
    \begin{itemize}
        \item 配置虚拟主机
        \item 设置Document Root为frontend/web和backend/web
        \item 配置URL重写规则
    \end{itemize}
    
    \item \textbf{验证部署}
    \begin{itemize}
        \item 访问前台首页
        \item 访问后台登录页
        \item 测试主要功能
    \end{itemize}
\end{enumerate}

\subsection{配置说明}

\subsubsection{数据库配置}
\begin{lstlisting}[language=PHP, basicstyle=\ttfamily\small]
// common/config/main-local.php
return [
    'components' => [
        'db' => [
            'class' => 'yii\db\Connection',
            'dsn' => 'mysql:host=localhost;dbname=news_system',
            'username' => 'root',
            'password' => '',
            'charset' => 'utf8mb4',
        ],
    ],
];
\end{lstlisting}

\subsubsection{URL配置}
\begin{lstlisting}[language=PHP, basicstyle=\ttfamily\small]
// frontend/config/main.php 或 backend/config/main.php
'components' => [
    'urlManager' => [
        'enablePrettyUrl' => true,
        'showScriptName' => false,
        'rules' => [],
    ],
],
\end{lstlisting}

\section{维护设计}

\subsection{日志管理}

\begin{itemize}
    \item \textbf{应用日志}:runtime/logs/app.log
    \item \textbf{错误日志}:runtime/logs/error.log
    \item \textbf{访问日志}:Web服务器日志
    \item 日志级别:error, warning, info, trace
    \item 日志轮转:按日期或文件大小
\end{itemize}

\subsection{数据备份}

\begin{itemize}
    \item \textbf{数据库备份}:
    \begin{itemize}
        \item 每日自动备份
        \item 备份文件命名:news\_system\_YYYYMMDD.sql
        \item 保留最近30天的备份
    \end{itemize}
    \item \textbf{代码备份}:Git版本控制
    \item \textbf{文件备份}:定期备份uploads目录
\end{itemize}

\subsection{监控告警}

\begin{itemize}
    \item 服务器性能监控(CPU、内存、磁盘)
    \item 应用错误监控(500错误、异常日志)
    \item 数据库性能监控(慢查询、连接数)
\end{itemize}

\section{总结}

本设计文档详细描述了"抗战胜利80周年纪念网站"的系统设计方案,包括:

\begin{enumerate}
    \item \textbf{系统架构设计}:三层架构 + MVC模式,前后台分离
    \item \textbf{数据库设计}:19张表,满足课程要求,ER图清晰
    \item \textbf{接口设计}:RESTful风格,前后台接口完整
    \item \textbf{界面设计}:抗战主题风格,响应式布局
    \item \textbf{安全性设计}:认证授权、CSRF/XSS/SQL注入防护
    \item \textbf{性能优化设计}:数据库、前端、代码三层优化
    \item \textbf{测试设计}:单元、功能、集成、安全、性能测试
    \item \textbf{部署设计}:详细部署步骤和配置说明
    \item \textbf{维护设计}:日志、备份、监控方案
\end{enumerate}

本设计方案技术先进、架构合理、安全可靠,为项目实施提供了完整的技术指导。

%--------------------------结束--------------------------------
\end{document}


