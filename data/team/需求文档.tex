\documentclass[a4paper]{article}

\input{style/ch_xelatex.tex}
\input{style/scala.tex}
\graphicspath{ {images/} }
\usepackage{ctex}
\setCJKmainfont{SimSun}[AutoFakeBold=2.5, AutoFakeSlant=0.3]
\setCJKsansfont{SimHei}[AutoFakeBold=2.5]
\setCJKmonofont{FangSong}
\usepackage{graphicx}
\usepackage{color,framed}
\usepackage{listings}
\usepackage{caption}
\usepackage{amssymb}
\usepackage{enumerate}
\usepackage{xcolor}
\usepackage{bm} 
\usepackage{lastpage}
\usepackage{fancyhdr}
\usepackage{tabularx}  
\usepackage{geometry}
\usepackage{float}
\usepackage{multirow}
\usepackage{booktabs}

\geometry{a4paper,left=2.3cm,right=2.3cm,top=2.7cm,bottom=2.7cm}

\usepackage{fancyhdr}
\pagestyle{fancy}
\lhead{\kaishu \leftmark}
\rhead{\kaishu 互联网数据库作业报告}
\lfoot{}
\cfoot{\thepage}
\rfoot{}
\renewcommand{\headrulewidth}{0.1pt}  
\renewcommand{\footrulewidth}{0pt}
\setlength{\textfloatsep}{10mm}

\begin{document}
\renewcommand{\contentsname}{目\ 录}
\renewcommand{\figurename}{图}
\renewcommand{\tablename}{表}
\renewcommand{\today}{\number\year 年 \number\month 月 \number\day 日}

%-------------------------封面----------------
\begin{titlepage}
    \begin{center}
    \includegraphics[width=0.8\textwidth]{NKU.png}\\[1cm]
    \vspace{20mm}
		\textbf{\huge\textbf{\kaishu{计算机学院}}}\\[0.5cm]
		\textbf{\huge{\kaishu{互联网数据库开发}}}\\[2.3cm]
		\textbf{\Huge\textbf{\kaishu{团队作业1——需求文档}}}
		\vspace{\fill}
    
    \centering
    \begin{tabular}{ll}
        \textsc{\LARGE \kaishu{项目名称\ :\ 抗战胜利80周年纪念网站}} \\[0.3cm]
        \textsc{\LARGE \kaishu{团队名称\ :\ 抗战纪念开发组}} \\[0.5cm]
        \textsc{\LARGE \kaishu{组长\ :\ 童汉鑫\ (学号:2311995)}} \\[0.3cm]
        \textsc{\LARGE \kaishu{组员\ :\ 刘浩泽\ (学号:2212478)}} \\[0.3cm]
        \textsc{\LARGE \kaishu{组员\ :\ 彭浩然\ (学号:2313314)}} \\[0.3cm]
        \textsc{\LARGE \kaishu{组员\ :\ 董珺\ (学号:2212880)}} \\[0.3cm]
    \end{tabular}
    \vfill
    {\Large \today}
    \end{center}
\end{titlepage}

\renewcommand {\thefigure}{\thesection{}.\arabic{figure}}
\renewcommand{\figurename}{图}
\renewcommand{\contentsname}{目录}  
\cfoot{\thepage\ of \pageref{LastPage}}

\clearpage
\tableofcontents
\newpage

%--------------------------正文内容--------------------------------

\section{项目概述}

\subsection{项目背景}
2025年是中国人民抗日战争暨世界反法西斯战争胜利80周年。为铭记历史、缅怀先烈、珍爱和平、开创未来,本项目旨在开发一个集历史展示、信息管理、互动交流于一体的综合性纪念网站。

该网站通过互联网数据库技术,系统化地展示抗战历史资料、英雄人物事迹、重大战役介绍、历史遗址信息等内容,为广大用户提供一个学习历史、表达敬意、交流感想的网络平台。

\subsection{项目目标}
\begin{enumerate}
    \item \textbf{历史传承}:系统化展示抗战历史资料,让更多人了解这段历史
    \item \textbf{信息管理}:建立完善的后台管理系统,实现内容的动态维护
    \item \textbf{用户互动}:提供评论、留言、点赞等功能,增强用户参与感
    \item \textbf{数据可视化}:通过图表展示访问统计、内容分布等数据
    \item \textbf{响应式设计}:适配多种设备,提供良好的用户体验
\end{enumerate}

\subsection{技术选型}
\begin{itemize}
    \item \textbf{开发框架}:Yii2 Advanced Template
    \item \textbf{数据库}:MySQL 5.7+
    \item \textbf{前端技术}:HTML5 + CSS3 + JavaScript + Bootstrap
    \item \textbf{图表库}:ECharts
    \item \textbf{开发环境}:XAMPP (Apache + PHP 8.2 + MySQL)
    \item \textbf{版本控制}:Git + GitHub
\end{itemize}

\section{用户需求分析}

\subsection{用户角色定义}
本系统共设计三类用户角色,每类角色具有不同的权限和功能需求:

\subsubsection{普通访客(游客)}
\begin{itemize}
    \item 无需注册即可访问网站
    \item 可浏览所有公开的历史资料、文章内容
    \item 可查看抗战历史时间线、英雄人物介绍
    \item 可在留言板发表留言(需填写昵称和邮箱)
    \item 无法进行点赞、评论等需要登录的操作
\end{itemize}

\subsubsection{注册用户}
\begin{itemize}
    \item 需要注册账号并登录
    \item 拥有普通访客的所有权限
    \item 可对文章进行点赞操作
    \item 可发表评论并与其他用户互动
    \item 可收藏感兴趣的文章内容
    \item 可查看个人浏览历史记录
\end{itemize}

\subsubsection{管理员}
\begin{itemize}
    \item 具有最高权限,可访问后台管理系统
    \item 可管理所有文章内容(增删改查)
    \item 可管理文章分类和标签
    \item 可管理用户账号(查看、禁用、删除)
    \item 可审核和管理用户评论
    \item 可查看和管理留言板内容
    \item 可查看网站统计数据和访问日志
    \item 可管理团队成员信息
\end{itemize}

\subsection{功能需求清单}

\subsubsection{前台功能需求}

\paragraph{首页模块}
\begin{enumerate}
    \item \textbf{Hero横幅展示}
    \begin{itemize}
        \item 显示主题标语:"铭记历史 珍爱和平"
        \item 显示纪念副标题:"纪念中国人民抗日战争暨世界反法西斯战争胜利80周年"
        \item 显示年份标识:1945-2025
    \end{itemize}
    
    \item \textbf{分类导航}
    \begin{itemize}
        \item 显示所有文章分类(重大战役、英雄人物、历史遗址等)
        \item 支持点击分类快速筛选对应内容
        \item 高亮显示当前选中的分类
    \end{itemize}
    
    \item \textbf{热门专题展示}
    \begin{itemize}
        \item 按点赞数或浏览量排序展示热门文章
        \item 每篇文章显示:标题、摘要、分类、作者、浏览量、点赞数
        \item 卡片式布局,配有相关主题图片
    \end{itemize}
    
    \item \textbf{最新发布展示}
    \begin{itemize}
        \item 按发布时间倒序展示最新文章
        \item 展示方式同热门专题
    \end{itemize}
    
    \item \textbf{抗战历史时间线}
    \begin{itemize}
        \item 时间轴形式展示重要历史事件
        \item 包含事件:九一八事变、七七事变、南京大屠杀、台儿庄大捷、百团大战、抗战胜利
        \item 每个事件显示日期、标题和简要说明
    \end{itemize}
    
    \item \textbf{数据统计图表}
    \begin{itemize}
        \item 近30天访问趋势折线图(使用ECharts)
        \item 分类内容分布饼图
        \item 图表支持交互式查看详细数据
    \end{itemize}
\end{enumerate}

\paragraph{文章列表模块}
\begin{enumerate}
    \item \textbf{文章筛选功能}
    \begin{itemize}
        \item 按分类筛选(重大战役、英雄人物、历史遗址等8个分类)
        \item 按关键词搜索文章标题和内容
        \item 筛选条件可组合使用
    \end{itemize}
    
    \item \textbf{文章排序功能}
    \begin{itemize}
        \item 按发布时间排序(最新优先)
        \item 按浏览量排序(最热门优先)
        \item 按点赞数排序(最受欢迎优先)
    \end{itemize}
    
    \item \textbf{文章列表展示}
    \begin{itemize}
        \item 网格卡片式布局
        \item 每个卡片显示:封面图、标题、摘要、分类、浏览量、点赞数
        \item 支持分页,每页显示12篇文章
        \item 点击卡片跳转到文章详情页
    \end{itemize}
\end{enumerate}

\paragraph{文章详情模块}
\begin{enumerate}
    \item \textbf{文章内容展示}
    \begin{itemize}
        \item 显示文章标题、发布时间、作者、分类
        \item 显示浏览量、点赞数统计
        \item 完整显示文章正文内容
        \item 支持富文本格式(标题、段落、列表等)
    \end{itemize}
    
    \item \textbf{阅读进度条}
    \begin{itemize}
        \item 页面顶部显示阅读进度
        \item 随滚动实时更新进度百分比
        \item 提升用户阅读体验
    \end{itemize}
    
    \item \textbf{浮动工具栏}
    \begin{itemize}
        \item 页面右侧固定位置显示
        \item 点赞按钮:点击后增加点赞数,同一会话只能点赞一次
        \item 分享按钮:打开分享弹窗,支持分享到微博或复制链接
        \item 返回顶部按钮:快速滚动到页面顶部
    \end{itemize}
    
    \item \textbf{相关文章推荐}
    \begin{itemize}
        \item 根据当前文章分类推荐同类文章
        \item 显示3-4篇相关文章
        \item 展示文章标题和摘要
    \end{itemize}
    
    \item \textbf{评论系统}
    \begin{itemize}
        \item 显示所有用户评论,按时间倒序
        \item 每条评论显示:用户昵称、发布时间、评论内容
        \item 注册用户可发表评论
        \item 支持评论的审核和删除(管理员权限)
    \end{itemize}
\end{enumerate}

\paragraph{团队展示模块}
\begin{enumerate}
    \item \textbf{团队信息}
    \begin{itemize}
        \item 显示团队名称、简介
        \item 显示项目开发背景和目标
    \end{itemize}
    
    \item \textbf{成员展示}
    \begin{itemize}
        \item 卡片式展示每位成员
        \item 显示成员姓名、角色、负责模块
        \item 显示成员照片或头像
        \item 显示成员邮箱(可选)
    \end{itemize}
    
    \item \textbf{分工说明}
    \begin{itemize}
        \item 表格形式展示详细分工
        \item 包含:成员姓名、学号、负责模块、工作量
    \end{itemize}
\end{enumerate}

\paragraph{留言板模块}
\begin{enumerate}
    \item \textbf{留言发布}
    \begin{itemize}
        \item 游客无需登录即可留言
        \item 必填项:昵称、留言内容
        \item 选填项:邮箱
        \item 支持富文本输入
        \item 留言提交后显示成功提示
    \end{itemize}
    
    \item \textbf{留言展示}
    \begin{itemize}
        \item 列表形式展示所有留言
        \item 显示留言者昵称、留言时间、留言内容
        \item 按时间倒序排列
        \item 支持分页显示
    \end{itemize}
\end{enumerate}

\paragraph{用户认证模块}
\begin{enumerate}
    \item \textbf{用户注册}
    \begin{itemize}
        \item 必填项:用户名、密码、确认密码、邮箱
        \item 用户名需唯一,长度4-20个字符
        \item 密码强度验证(至少6位)
        \item 邮箱格式验证
        \item 注册成功后自动登录
    \end{itemize}
    
    \item \textbf{用户登录}
    \begin{itemize}
        \item 支持用户名或邮箱登录
        \item 记住登录状态(可选)
        \item 登录失败提示错误信息
        \item 登录成功跳转到之前访问的页面
    \end{itemize}
    
    \item \textbf{用户注销}
    \begin{itemize}
        \item 退出当前登录状态
        \item 清除session信息
        \item 跳转到首页
    \end{itemize}
\end{enumerate}

\subsubsection{后台功能需求}

\paragraph{文章管理模块}
\begin{enumerate}
    \item \textbf{文章列表}
    \begin{itemize}
        \item 表格形式展示所有文章
        \item 显示字段:ID、标题、分类、状态、浏览量、点赞数、发布时间
        \item 支持关键词搜索(标题)
        \item 支持分类筛选
        \item 支持状态筛选(已发布、草稿、已下架)
        \item 支持排序(按时间、浏览量、点赞数)
        \item 支持分页(每页10条)
    \end{itemize}
    
    \item \textbf{文章新增}
    \begin{itemize}
        \item 填写文章标题(必填)
        \item 选择文章分类(必填)
        \item 填写文章摘要(选填)
        \item 编辑文章正文(必填,支持富文本)
        \item 填写文章来源(选填)
        \item 填写作者(选填)
        \item 选择文章状态(已发布/草稿/已下架)
        \item 设置是否置顶
        \item 设置是否热门
        \item 保存后返回文章列表
    \end{itemize}
    
    \item \textbf{文章编辑}
    \begin{itemize}
        \item 加载现有文章数据
        \item 可修改所有文章字段
        \item 保存后更新数据库
        \item 记录修改时间
    \end{itemize}
    
    \item \textbf{文章删除}
    \begin{itemize}
        \item 单篇文章删除
        \item 批量删除功能
        \item 删除前二次确认
        \item 删除后无法恢复
    \end{itemize}
    
    \item \textbf{批量操作}
    \begin{itemize}
        \item 批量设置置顶/取消置顶
        \item 批量设置热门/取消热门
        \item 批量删除文章
        \item 支持多选功能
    \end{itemize}
    
    \item \textbf{快捷操作}
    \begin{itemize}
        \item 单篇文章一键置顶/取消置顶
        \item 单篇文章一键热门/取消热门
        \item 快速查看文章详情
    \end{itemize}
\end{enumerate}

\paragraph{分类管理模块}
\begin{enumerate}
    \item \textbf{分类列表}
    \begin{itemize}
        \item 表格展示所有分类
        \item 显示字段:ID、分类名称、描述、排序、状态、创建时间
        \item 支持搜索分类名称
        \item 显示每个分类下的文章数量
    \end{itemize}
    
    \item \textbf{分类新增}
    \begin{itemize}
        \item 填写分类名称(必填,唯一)
        \item 填写分类描述(选填)
        \item 设置排序顺序(数字,越小越靠前)
        \item 设置分类状态(启用/禁用)
    \end{itemize}
    
    \item \textbf{分类编辑}
    \begin{itemize}
        \item 修改分类信息
        \item 更新修改时间
    \end{itemize}
    
    \item \textbf{分类删除}
    \begin{itemize}
        \item 检查是否有文章使用该分类
        \item 如有关联文章则禁止删除
        \item 无关联文章可直接删除
    \end{itemize}
\end{enumerate}

\paragraph{用户管理模块}
\begin{enumerate}
    \item \textbf{用户列表}
    \begin{itemize}
        \item 表格展示所有用户
        \item 显示字段:用户ID、用户名、邮箱、角色、状态、注册时间
        \item 支持按用户名搜索
        \item 支持按角色筛选
        \item 支持按状态筛选
    \end{itemize}
    
    \item \textbf{用户新增}
    \begin{itemize}
        \item 填写用户名(必填,唯一)
        \item 填写密码(必填)
        \item 填写邮箱(必填,格式验证)
        \item 填写手机号(选填)
        \item 选择用户角色(普通用户/管理员)
        \item 设置用户状态(正常/禁用)
    \end{itemize}
    
    \item \textbf{用户编辑}
    \begin{itemize}
        \item 修改用户信息
        \item 修改用户角色
        \item 修改用户状态
        \item 重置用户密码
    \end{itemize}
    
    \item \textbf{用户删除}
    \begin{itemize}
        \item 检查用户是否有关联数据(文章、评论等)
        \item 删除前二次确认
        \item 删除用户及其关联数据
    \end{itemize}
\end{enumerate}

\paragraph{评论管理模块}
\begin{enumerate}
    \item \textbf{评论列表}
    \begin{itemize}
        \item 表格展示所有评论
        \item 显示字段:ID、文章标题、评论者、评论内容、状态、时间
        \item 支持按文章筛选
        \item 支持按状态筛选(待审核/已通过/已拒绝)
        \item 支持搜索评论内容
    \end{itemize}
    
    \item \textbf{评论审核}
    \begin{itemize}
        \item 查看评论详情
        \item 通过评论(显示在前台)
        \item 拒绝评论(不显示)
        \item 批量审核功能
    \end{itemize}
    
    \item \textbf{评论删除}
    \begin{itemize}
        \item 单条删除
        \item 批量删除
        \item 删除前确认
    \end{itemize}
\end{enumerate}

\paragraph{留言管理模块}
\begin{enumerate}
    \item \textbf{留言列表}
    \begin{itemize}
        \item 表格展示所有留言
        \item 显示字段:ID、留言者昵称、邮箱、留言内容、IP地址、是否已读、时间
        \item 支持按已读/未读筛选
        \item 支持搜索留言内容
        \item 未读留言高亮显示
    \end{itemize}
    
    \item \textbf{留言处理}
    \begin{itemize}
        \item 标记为已读
        \item 批量标记已读
        \item 查看留言详情
        \item 回复留言(发送邮件)
    \end{itemize}
    
    \item \textbf{留言删除}
    \begin{itemize}
        \item 单条删除
        \item 批量删除
        \item 删除前确认
    \end{itemize}
\end{enumerate}

\paragraph{团队管理模块}
\begin{enumerate}
    \item \textbf{成员管理}
    \begin{itemize}
        \item 查看团队成员列表
        \item 新增团队成员信息
        \item 编辑成员信息(姓名、角色、邮箱、简介等)
        \item 删除团队成员
        \item 上传成员照片
    \end{itemize}
    
    \item \textbf{部门管理}
    \begin{itemize}
        \item 管理团队部门信息
        \item 设置部门名称和描述
        \item 关联部门与成员
    \end{itemize}
\end{enumerate}

\paragraph{统计分析模块}
\begin{enumerate}
    \item \textbf{数据概览}
    \begin{itemize}
        \item 显示文章总数、用户总数、评论总数、留言总数
        \item 显示今日新增数据统计
        \item 显示本周/本月数据趋势
    \end{itemize}
    
    \item \textbf{访问统计}
    \begin{itemize}
        \item 折线图展示近30天访问趋势
        \item 显示访问量、浏览量、独立访客数
        \item 支持按日期范围筛选
    \end{itemize}
    
    \item \textbf{内容统计}
    \begin{itemize}
        \item 饼图展示分类内容分布
        \item 柱状图展示文章发布趋势
        \item 排行榜(热门文章、活跃用户)
    \end{itemize}
\end{enumerate}

\subsection{非功能性需求}

\subsubsection{性能需求}
\begin{enumerate}
    \item \textbf{响应时间}
    \begin{itemize}
        \item 页面首次加载时间 < 3秒
        \item 普通操作响应时间 < 1秒
        \item 数据库查询响应时间 < 500ms
    \end{itemize}
    
    \item \textbf{并发能力}
    \begin{itemize}
        \item 支持至少100个并发用户同时访问
        \item 数据库连接池配置合理
    \end{itemize}
    
    \item \textbf{数据量}
    \begin{itemize}
        \item 支持至少500篇文章
        \item 支持至少1000条评论
        \item 支持至少100个注册用户
    \end{itemize}
\end{enumerate}

\subsubsection{安全需求}
\begin{enumerate}
    \item \textbf{用户认证}
    \begin{itemize}
        \item 密码加密存储(使用Yii2的security组件)
        \item Session管理,防止会话劫持
        \item 登录失败次数限制
    \end{itemize}
    
    \item \textbf{权限控制}
    \begin{itemize}
        \item 严格的角色权限划分
        \item 后台操作需要管理员权限
        \item 防止越权访问
    \end{itemize}
    
    \item \textbf{数据安全}
    \begin{itemize}
        \item SQL注入防护(使用参数化查询)
        \item XSS攻击防护(输入过滤、输出转义)
        \item CSRF攻击防护(使用CSRF Token)
        \item 敏感信息加密存储
    \end{itemize}
    
    \item \textbf{输入验证}
    \begin{itemize}
        \item 前端表单验证
        \item 后端数据验证
        \item 文件上传类型和大小限制
    \end{itemize}
\end{enumerate}

\subsubsection{可用性需求}
\begin{enumerate}
    \item \textbf{界面设计}
    \begin{itemize}
        \item 符合抗战纪念主题的视觉风格(深红、金色配色)
        \item 布局清晰,操作直观
        \item 响应式设计,适配PC、平板、手机
    \end{itemize}
    
    \item \textbf{用户体验}
    \begin{itemize}
        \item 操作反馈及时(成功/失败提示)
        \item 错误提示友好明确
        \item 表单验证提示清晰
        \item 加载状态提示
    \end{itemize}
    
    \item \textbf{浏览器兼容}
    \begin{itemize}
        \item 支持Chrome、Firefox、Edge等主流浏览器
        \item 浏览器版本:最近两个主要版本
    \end{itemize}
\end{enumerate}

\subsubsection{可维护性需求}
\begin{enumerate}
    \item \textbf{代码规范}
    \begin{itemize}
        \item 遵循PSR编码规范
        \item 统一的命名规范
        \item 每个PHP文件包含注释头(作者、日期、描述)
    \end{itemize}
    
    \item \textbf{文档要求}
    \begin{itemize}
        \item 关键功能添加代码注释
        \item 数据库表和字段添加注释
        \item 提供完整的开发文档
    \end{itemize}
    
    \item \textbf{模块化设计}
    \begin{itemize}
        \item MVC架构清晰
        \item 模块间低耦合
        \item 代码复用性高
    \end{itemize}
\end{enumerate}

\subsubsection{可扩展性需求}
\begin{enumerate}
    \item \textbf{功能扩展}
    \begin{itemize}
        \item 预留用户个人中心功能接口
        \item 预留文章附件上传功能
        \item 预留多语言支持
    \end{itemize}
    
    \item \textbf{数据扩展}
    \begin{itemize}
        \item 数据库设计支持未来扩展
        \item 表结构合理,减少后期修改
    \end{itemize}
\end{enumerate}

\section{数据库需求}

\subsection{数据实体分析}
本系统涉及以下主要数据实体:

\begin{enumerate}
    \item \textbf{用户(User)}:存储系统用户信息
    \item \textbf{文章(Article)}:存储历史资料文章
    \item \textbf{分类(Category)}:存储文章分类信息
    \item \textbf{评论(Comment)}:存储用户评论
    \item \textbf{留言(GuestBook)}:存储访客留言
    \item \textbf{标签(Tag)}:存储文章标签
    \item \textbf{团队成员(TeamMember)}:存储团队成员信息
    \item \textbf{统计数据(Stats)}:存储访问统计信息
\end{enumerate}

\subsection{数据表设计要求}
根据课程要求,本项目数据库表数量需 \textbf{> 10张表},当前设计共计 \textbf{19张表},具体如下:

\begin{table}[H]
\centering
\caption{数据库表列表}
\begin{tabularx}{\textwidth}{|l|X|l|}
\hline
\textbf{序号} & \textbf{表名} & \textbf{说明} \\ \hline
1 & pre\_sys\_user & 系统用户表 \\ \hline
2 & pre\_news\_category & 文章分类表 \\ \hline
3 & pre\_news\_article & 文章主表 \\ \hline
4 & pre\_news\_comment & 文章评论表 \\ \hline
5 & pre\_guestbook & 留言板表 \\ \hline
6 & pre\_article\_tag & 文章标签表 \\ \hline
7 & pre\_article\_tag\_relation & 文章标签关联表 \\ \hline
8 & pre\_article\_view\_log & 文章浏览日志表 \\ \hline
9 & pre\_article\_like & 文章点赞记录表 \\ \hline
10 & pre\_article\_favorite & 文章收藏表 \\ \hline
11 & pre\_article\_attachment & 文章附件表 \\ \hline
12 & pre\_team\_department & 团队部门表 \\ \hline
13 & pre\_team\_member & 团队成员表 \\ \hline
14 & pre\_daily\_stats & 每日统计表 \\ \hline
15 & pre\_category\_stats & 分类统计表 \\ \hline
16 & pre\_link & 友情链接表 \\ \hline
17 & pre\_tag & 标签表 \\ \hline
18 & pre\_visit\_log & 访问日志表 \\ \hline
19 & pre\_config & 系统配置表 \\ \hline
\end{tabularx}
\end{table}

\subsection{核心数据表结构}

\subsubsection{用户表(pre\_sys\_user)}
\begin{table}[H]
\centering
\caption{用户表结构}
\begin{tabularx}{\textwidth}{|l|l|l|X|}
\hline
\textbf{字段名} & \textbf{类型} & \textbf{约束} & \textbf{说明} \\ \hline
uid & INT & PRIMARY KEY, AUTO\_INCREMENT & 用户ID \\ \hline
username & VARCHAR(50) & NOT NULL, UNIQUE & 用户名 \\ \hline
password\_hash & VARCHAR(255) & NOT NULL & 密码哈希 \\ \hline
email & VARCHAR(100) & NULL & 邮箱 \\ \hline
phone & VARCHAR(20) & NULL & 手机号 \\ \hline
avatar & VARCHAR(255) & NULL & 头像 \\ \hline
role & TINYINT & DEFAULT 0 & 角色:0普通用户 1管理员 \\ \hline
status & TINYINT & DEFAULT 1 & 状态:0禁用 1正常 \\ \hline
created\_at & DATETIME & DEFAULT CURRENT\_TIMESTAMP & 创建时间 \\ \hline
updated\_at & DATETIME & DEFAULT CURRENT\_TIMESTAMP & 更新时间 \\ \hline
\end{tabularx}
\end{table}

\subsubsection{文章分类表(pre\_news\_category)}
\begin{table}[H]
\centering
\caption{文章分类表结构}
\begin{tabularx}{\textwidth}{|l|l|l|X|}
\hline
\textbf{字段名} & \textbf{类型} & \textbf{约束} & \textbf{说明} \\ \hline
cid & INT & PRIMARY KEY, AUTO\_INCREMENT & 分类ID \\ \hline
name & VARCHAR(50) & NOT NULL, UNIQUE & 分类名称 \\ \hline
description & TEXT & NULL & 分类描述 \\ \hline
sort & INT & DEFAULT 0 & 排序 \\ \hline
status & TINYINT & DEFAULT 1 & 状态:0禁用 1启用 \\ \hline
created\_at & DATETIME & DEFAULT CURRENT\_TIMESTAMP & 创建时间 \\ \hline
updated\_at & DATETIME & DEFAULT CURRENT\_TIMESTAMP & 更新时间 \\ \hline
\end{tabularx}
\end{table}

\subsubsection{文章主表(pre\_news\_article)}
\begin{table}[H]
\centering
\caption{文章主表结构}
\begin{tabularx}{\textwidth}{|l|l|l|X|}
\hline
\textbf{字段名} & \textbf{类型} & \textbf{约束} & \textbf{说明} \\ \hline
aid & INT & PRIMARY KEY, AUTO\_INCREMENT & 文章ID \\ \hline
cid & INT & NOT NULL, FOREIGN KEY & 分类ID \\ \hline
title & VARCHAR(200) & NOT NULL & 文章标题 \\ \hline
summary & TEXT & NULL & 文章摘要 \\ \hline
content & LONGTEXT & NOT NULL & 文章内容 \\ \hline
author & VARCHAR(50) & NULL & 作者 \\ \hline
source & VARCHAR(100) & NULL & 来源 \\ \hline
views & INT & DEFAULT 0 & 浏览量 \\ \hline
likes & INT & DEFAULT 0 & 点赞数 \\ \hline
is\_top & TINYINT & DEFAULT 0 & 是否置顶 \\ \hline
is\_hot & TINYINT & DEFAULT 0 & 是否热门 \\ \hline
status & TINYINT & DEFAULT 1 & 状态:0草稿 1已发布 2已下架 \\ \hline
created\_at & DATETIME & DEFAULT CURRENT\_TIMESTAMP & 创建时间 \\ \hline
updated\_at & DATETIME & DEFAULT CURRENT\_TIMESTAMP & 更新时间 \\ \hline
\end{tabularx}
\end{table}

\subsubsection{评论表(pre\_news\_comment)}
\begin{table}[H]
\centering
\caption{评论表结构}
\begin{tabularx}{\textwidth}{|l|l|l|X|}
\hline
\textbf{字段名} & \textbf{类型} & \textbf{约束} & \textbf{说明} \\ \hline
comment\_id & INT & PRIMARY KEY, AUTO\_INCREMENT & 评论ID \\ \hline
aid & INT & NOT NULL, FOREIGN KEY & 文章ID \\ \hline
uid & INT & NOT NULL, FOREIGN KEY & 用户ID \\ \hline
content & TEXT & NOT NULL & 评论内容 \\ \hline
parent\_id & INT & NULL & 父评论ID(用于回复) \\ \hline
status & TINYINT & DEFAULT 0 & 状态:0待审核 1已通过 2已拒绝 \\ \hline
created\_at & DATETIME & DEFAULT CURRENT\_TIMESTAMP & 创建时间 \\ \hline
updated\_at & DATETIME & DEFAULT CURRENT\_TIMESTAMP & 更新时间 \\ \hline
\end{tabularx}
\end{table}

\subsection{数据字典要求}
\begin{itemize}
    \item 所有表名使用 \texttt{pre\_} 前缀
    \item 主键统一命名为表名简写 + \texttt{id} 或特定ID名称
    \item 时间字段统一使用 \texttt{created\_at}、\texttt{updated\_at}
    \item 状态字段统一使用 \texttt{status},TINYINT类型
    \item 所有表和字段添加中文注释
    \item 外键关联关系明确,保证数据完整性
\end{itemize}

\section{系统接口需求}

\subsection{前后端交互接口}

\subsubsection{文章接口}
\begin{table}[H]
\centering
\caption{文章相关接口}
\begin{tabularx}{\textwidth}{|l|l|X|}
\hline
\textbf{接口名称} & \textbf{请求方式} & \textbf{功能说明} \\ \hline
/site/index & GET & 获取首页数据(热门文章、最新文章) \\ \hline
/site/news & GET & 获取文章列表(支持分类、排序、分页) \\ \hline
/site/view & GET & 获取文章详情(id参数) \\ \hline
/site/like & POST & 文章点赞接口(id参数) \\ \hline
\end{tabularx}
\end{table}

\subsubsection{评论接口}
\begin{table}[H]
\centering
\caption{评论相关接口}
\begin{tabularx}{\textwidth}{|l|l|X|}
\hline
\textbf{接口名称} & \textbf{请求方式} & \textbf{功能说明} \\ \hline
/site/comment & POST & 提交评论(需登录) \\ \hline
/comment/index & GET & 获取评论列表(后台) \\ \hline
/comment/update & POST & 审核评论(后台) \\ \hline
\end{tabularx}
\end{table}

\subsubsection{用户接口}
\begin{table}[H]
\centering
\caption{用户相关接口}
\begin{tabularx}{\textwidth}{|l|l|X|}
\hline
\textbf{接口名称} & \textbf{请求方式} & \textbf{功能说明} \\ \hline
/site/signup & GET/POST & 用户注册 \\ \hline
/site/login & GET/POST & 用户登录 \\ \hline
/site/logout & POST & 用户注销 \\ \hline
\end{tabularx}
\end{table}

\subsection{第三方接口}
\begin{enumerate}
    \item \textbf{图表库(ECharts)}
    \begin{itemize}
        \item CDN引入:https://cdn.jsdelivr.net/npm/echarts@5.4.3/dist/echarts.min.js
        \item 用于数据可视化展示
    \end{itemize}
    
    \item \textbf{图标库(Font Awesome)}
    \begin{itemize}
        \item CDN引入:https://cdnjs.cloudflare.com/ajax/libs/font-awesome/6.4.0/css/all.min.css
        \item 用于界面图标显示
    \end{itemize}
\end{enumerate}

\section{运行环境需求}

\subsection{服务器环境}
\begin{itemize}
    \item \textbf{操作系统}:Windows 10/11 或 Linux (Ubuntu 20.04+)
    \item \textbf{Web服务器}:Apache 2.4+
    \item \textbf{PHP版本}:PHP 8.0+
    \item \textbf{数据库}:MySQL 5.7+ 或 MariaDB 10.3+
\end{itemize}

\subsection{客户端环境}
\begin{itemize}
    \item \textbf{浏览器}:Chrome 90+、Firefox 88+、Edge 90+、Safari 14+
    \item \textbf{屏幕分辨率}:最低 1280×720,推荐 1920×1080
    \item \textbf{网络环境}:稳定的互联网连接
\end{itemize}

\subsection{开发环境}
\begin{itemize}
    \item \textbf{开发工具}:Visual Studio Code / PhpStorm
    \item \textbf{版本控制}:Git 2.30+
    \item \textbf{包管理器}:Composer 2.0+
    \item \textbf{集成环境}:XAMPP 8.2.12(包含Apache + MySQL + PHP)
\end{itemize}

\section{项目约束与假设}

\subsection{项目约束}
\begin{enumerate}
    \item \textbf{时间约束}
    \begin{itemize}
        \item 项目开发周期:2025年10月 - 2025年12月
        \item 代码提交截止日期:2025年12月24日
        \item 不可延期交付
    \end{itemize}
    
    \item \textbf{技术约束}
    \begin{itemize}
        \item 必须使用Yii2框架开发
        \item 必须使用MySQL数据库
        \item 必须遵循MVC架构
        \item 数据库表数量 > 10张
    \end{itemize}
    
    \item \textbf{人力约束}
    \begin{itemize}
        \item 团队人数:4人
        \item 每位成员至少编写MVC三层各一个文件
        \item 需明确分工和工作量分配
    \end{itemize}
    
    \item \textbf{文档约束}
    \begin{itemize}
        \item 需提供7个团队文档(需求、设计、实现、用户手册、部署、PPT、录屏)
        \item 需提供3个个人作业文档
        \item 所有文档格式和命名需符合规范
    \end{itemize}
\end{enumerate}

\subsection{项目假设}
\begin{enumerate}
    \item 用户具备基本的计算机和互联网使用能力
    \item 服务器环境配置正确,满足系统运行要求
    \item 项目部署在局域网或公网环境,网络连接稳定
    \item 历史资料内容真实可靠,来源合法
    \item 用户注册时提供的信息真实有效
\end{enumerate}

\section{需求优先级}

根据功能重要性和实现难度,将需求划分为三个优先级:

\subsection{高优先级(必须实现)}
\begin{enumerate}
    \item 用户注册、登录、注销功能
    \item 文章的增删改查(前后台)
    \item 文章分类管理
    \item 文章列表展示与筛选
    \item 文章详情页展示
    \item 后台管理系统基本框架
    \item 数据库设计与实现(>10张表)
\end{enumerate}

\subsection{中优先级(重要功能)}
\begin{enumerate}
    \item 文章点赞功能
    \item 评论系统(发表、审核、管理)
    \item 留言板功能
    \item 用户管理(后台)
    \item 首页数据统计图表
    \item 阅读进度条
    \item 文章分享功能
    \item 返回顶部功能
\end{enumerate}

\subsection{低优先级(优化功能)}
\begin{enumerate}
    \item 文章附件上传
    \item 用户个人中心
    \item 文章收藏功能
    \item 高级搜索功能
    \item 多语言支持
    \item 邮件通知功能
    \item 一键部署脚本
\end{enumerate}

\section{验收标准}

\subsection{功能验收}
\begin{enumerate}
    \item 所有高优先级功能全部实现且运行正常
    \item 中优先级功能至少实现80\%
    \item 前后台界面完整,无明显布局错误
    \item 数据库表数量达到要求(>10张)
    \item 每位成员均编写了MVC三层文件
\end{enumerate}

\subsection{性能验收}
\begin{enumerate}
    \item 页面加载时间符合要求(<3秒)
    \item 无明显的性能瓶颈
    \item 支持多用户并发访问
\end{enumerate}

\subsection{安全验收}
\begin{enumerate}
    \item 通过SQL注入测试
    \item 通过XSS攻击测试
    \item 通过CSRF攻击测试
    \item 密码加密存储
    \item 权限控制有效
\end{enumerate}

\subsection{代码质量验收}
\begin{enumerate}
    \item 代码符合PSR规范
    \item 关键代码添加注释
    \item 数据库表和字段有注释
    \item MVC架构清晰
    \item 无严重的代码冗余
\end{enumerate}

\subsection{文档验收}
\begin{enumerate}
    \item 7个团队文档齐全(需求、设计、实现、用户手册、部署、PPT、录屏)
    \item 3个个人作业文档齐全
    \item 文档格式规范,命名正确
    \item 文档内容完整、清晰
\end{enumerate}

\section{风险分析}

\subsection{技术风险}
\begin{table}[H]
\centering
\caption{技术风险分析}
\begin{tabularx}{\textwidth}{|l|X|X|l|}
\hline
\textbf{风险} & \textbf{影响} & \textbf{应对措施} & \textbf{优先级} \\ \hline
Yii2框架不熟悉 & 开发进度延迟 & 提前学习官方文档,参考示例项目 & 高 \\ \hline
数据库设计不合理 & 后期修改困难 & 前期充分讨论,绘制ER图 & 高 \\ \hline
前后端集成问题 & 功能无法正常使用 & 定义清晰的接口规范 & 中 \\ \hline
浏览器兼容性 & 部分用户无法访问 & 使用标准Web技术,多浏览器测试 & 中 \\ \hline
\end{tabularx}
\end{table}

\subsection{项目管理风险}
\begin{table}[H]
\centering
\caption{项目管理风险分析}
\begin{tabularx}{\textwidth}{|l|X|X|l|}
\hline
\textbf{风险} & \textbf{影响} & \textbf{应对措施} & \textbf{优先级} \\ \hline
任务分工不明确 & 重复工作或遗漏 & 明确分工,定期同步进度 & 高 \\ \hline
成员沟通不畅 & 开发进度受阻 & 建立微信群,使用Git协作 & 高 \\ \hline
时间管理不当 & 无法按时交付 & 制定详细计划,设置里程碑 & 高 \\ \hline
代码冲突频繁 & 影响开发效率 & 遵循Git工作流,及时解决冲突 & 中 \\ \hline
\end{tabularx}
\end{table}

\subsection{需求变更风险}
\begin{table}[H]
\centering
\caption{需求变更风险分析}
\begin{tabularx}{\textwidth}{|l|X|X|l|}
\hline
\textbf{风险} & \textbf{影响} & \textbf{应对措施} & \textbf{优先级} \\ \hline
需求理解偏差 & 返工重做 & 与老师确认需求细节 & 高 \\ \hline
需求频繁变更 & 无法按时完成 & 冻结核心需求,变更需评估 & 中 \\ \hline
功能扩展过多 & 超出能力范围 & 按优先级实现,保证核心功能 & 中 \\ \hline
\end{tabularx}
\end{table}

\section{项目团队与分工}

\subsection{团队成员}
\begin{table}[H]
\centering
\caption{团队成员信息}
\begin{tabularx}{\textwidth}{|l|l|l|X|}
\hline
\textbf{姓名} & \textbf{学号} & \textbf{角色} & \textbf{负责模块} \\ \hline
童汉鑫 & 2311995 & 组长 & 评论管理、团队展示、成员管理、部署文档 \\ \hline
刘浩泽 & 2212478 & 组员 & 首页、文章管理、分类管理、爬虫、需求文档、设计文档 \\ \hline
彭浩然 & 2313314 & 组员 & 登录注册、留言、实现文档、用户手册 \\ \hline
董珺 & 2212880 & 组员 & 留言板、团队展示 \\ \hline
\end{tabularx}
\end{table}

\textbf{GitHub 仓库}:\url{https://github.com/lhz191/InternetDatabaseDevelopment2025}

\subsection{详细分工}
\begin{table}[H]
\centering
\caption{模块分工详情}
\begin{tabularx}{\textwidth}{|l|X|l|}
\hline
\textbf{成员} & \textbf{负责内容} & \textbf{工作量占比} \\ \hline
童汉鑫 & 
\begin{itemize}
    \item 前台评论功能
    \item 后台评论审核与管理
    \item 文章详情页优化
    \item 团队展示页面优化
    \item 部署文档编写
\end{itemize} & 20\% \\ \hline
刘浩泽 & 
\begin{itemize}
    \item 前台首页设计与实现
    \item 后台文章管理(CRUD)
    \item 后台分类管理(CRUD)
    \item 文章筛选、排序功能
    \item Python爬虫开发(生成500+文章)
    \item 动态图表开发(ECharts)
    \item 数据库表设计(19张表)
    \item 需求文档编写
    \item 设计文档编写
\end{itemize} & 35\% \\ \hline
彭浩然 & 
\begin{itemize}
    \item 用户注册、登录、注销功能
    \item 后台用户管理(CRUD)
    \item 权限控制
    \item 前台留言功能
    \item 实现文档编写
    \item 用户手册编写
\end{itemize} & 25\% \\ \hline
董珺 & 
\begin{itemize}
    \item 前台留言板功能
    \item 后台留言管理
    \item 前台团队展示页面
    \item 后台团队成员管理
\end{itemize} & 20\% \\ \hline
\end{tabularx}
\end{table}

\subsection{开发计划}
\begin{table}[H]
\centering
\caption{项目开发时间表}
\begin{tabularx}{\textwidth}{|l|X|l|}
\hline
\textbf{阶段} & \textbf{主要任务} & \textbf{时间} \\ \hline
需求分析 & 需求调研、需求文档编写 & 第1-2周 \\ \hline
系统设计 & 数据库设计、接口设计、设计文档编写 & 第3-4周 \\ \hline
开发实现 & 前后端功能开发、单元测试 & 第5-10周 \\ \hline
集成测试 & 功能集成、系统测试、Bug修复 & 第11周 \\ \hline
文档编写 & 用户手册、部署文档、PPT制作 & 第12周 \\ \hline
项目提交 & 录屏讲解、代码整理、最终提交 & 第12周末 \\ \hline
\end{tabularx}
\end{table}

\section{总结}

本需求文档详细描述了"抗战胜利80周年纪念网站"项目的各项需求,包括:

\begin{enumerate}
    \item \textbf{功能需求}:涵盖前台展示、后台管理、用户认证等核心功能
    \item \textbf{非功能需求}:明确了性能、安全、可用性、可维护性等质量要求
    \item \textbf{数据库需求}:设计了19张表,满足课程要求(>10张表)
    \item \textbf{接口需求}:定义了前后端交互接口和第三方接口
    \item \textbf{运行环境需求}:明确了服务器、客户端、开发环境要求
    \item \textbf{项目约束与假设}:列出了项目限制条件和前提假设
    \item \textbf{需求优先级}:划分了高、中、低三个优先级
    \item \textbf{验收标准}:明确了功能、性能、安全、代码质量、文档等验收标准
    \item \textbf{风险分析}:识别了技术、管理、需求变更等风险及应对措施
    \item \textbf{项目团队与分工}:明确了成员分工和开发计划
\end{enumerate}

本项目将严格按照需求文档执行,确保按时交付高质量的系统,并提供完整的技术文档。通过本项目,团队成员将深入学习Yii2框架、MySQL数据库、Web前端技术等知识,提升实际开发能力和团队协作能力。

%--------------------------结束--------------------------------
\end{document}

